\section{Discussion}
\subsection{Modular systems did not gain dominance on survivability}
Greedy methodologies, including elitism and tournament selection scheme, impede the emergence of modularity under our evolutionary simulations. This implies that individuals who performed optimally in the early stage might not be optimal on modularity. In other words, the most competitive elites in each generation did not have the most modular gene regulatory networks. 

Overall, these phenomena suggest that the modularity emergence condition, namely gene specialization promotes modular networks, may not be plausible to explain biological modularity. They indicated that modules in the simulated gene regulatory networks did not gain dominance in determining the survivability of individuals. However, biologically, modular networks are dominant and ubiquitous. In order to further investigate the plausibility of this theory, namely specialization driving modularity, we obtained the most optimal gene regulatory network among networks that were the most modular. Conversely, we also collected the network that was the least modular among those that had the greatest fitness value. These networks were collected from the generated results of experiments in Section X.X.

Biologically, I expected the fitness value of the latter would be lower than the fitness of the former. Nevertheless, the situation was converse. That is, some less modular networks were more robust than more modular ones, as Table X.X indicates. This is not consistent with what has been observed in biology.

Initially, I hypothesized that the inconsistency was due to the targeted gene activity patterns being over-simple. That is, the number of genes in a pattern was not sufficient or the number of patterns was not enough. A modular network may give great performance on complex tasks, but worse than non-modular ones for simple tasks. Thus, I conducted a complicated evolutionary simulation consisting $7$ patterns, each of which comprised 15 gene nodes. This evolution lasted for 35,000 generations. Other detailed parameters are in the Appendix X.

The evolutionary progress for the best-fit individual in every generation is in Figure 4.1. I conducted the modularity dominance analysis again and the results are in Table 4.4. Overall, the complex of gene activity patterns could not resolve the issue of non-dominance for modular networks on survivability. 

\subsection{Inter-Module Connections Can Hamper Network Fitness}
Fitness values of gene regulatory networks were measured after removing interconnections between modules in order to understand the functionality of inter-module interactions. The results indicated that among 40 networks which had the highest fitness values and relatively low modularity Q scores in their corresponding evolutionary simulations, 24 of them demonstrated higher fitness after manually converting them into modular structures by deleting inter-module edges. That is, there existed non-modular networks that exhibited better fitness performance after removing all the inter-module connections. For example, the right network in Figure X.X was the consequence of removing inter-module connections of the network in the left. The fitness value of the latter was 0.9502 after it had removed 6\% connections of the former, whose fitness was 0.9472. Further statistical investigations will be conducted in the future.

Originally, we suspected that this deviance was due to the fact that these modified solutions had a lower density than was expected from the evolutionary operations (sec 2.4), and thus may have been excluded from the search space. Nevertheless, further investigation revealed that the average number of edges for those networks that increased fitness values after triming their inter-module connections was approximately 30. That is, it was not due to the bias on the sparseness that caused this anomaly. 

In order to further comprehend this phenomenon on why our evolutionary simulations could not find a path to the trimed networks, we recorded the fitness value of removing one inter-module edge in turn, until deleting all of them. We plotted graphs as Figure X.X, where x-axis represents the number of inter-module edges that have been discarded, y-axis represents the corresponding fitness values. Interestingly, most of our collected plots demonstrated a steady increasing trend for fitness vs deleting edge numbers, whereas genetic algorithms could not find these paths. 

\subsection{Fluctuant  landscapes are essential for generating modularity}
The stochastic fitness evaluation used by Espinosa-Soto and Wagner \cite{espinosa2010specialization} demonstrated much higher fitness and modularity Q score than Larson et al.'s deterministic fitness evaluation. Therefore, we hypothesized that a fluctuannt landscapes for individuals during the evolution might be necessary to develop high modularity. In order to verify this hypothesis, we collected 

\subsection{More modular networks require fewer connections}