\section{Experiments}
Gene activity patterns and the essential parameters of our evolutionary simulations are provided in the form of Tables 3.1 and 3.2 in order to facilitate repeatability of these experiments. The detailed explanations of these parameters are given after Table 3.2. Overall, only the elite number will be specified in each experiment, since only that may vary in different experiments. All the other parameters are specified in Table 3.1 and Table 3.2 and are consistent in the experiments. 
\begin{table}[h!]
	\centering
	\caption{Table to test captions and labels}
	\label{table:X}
	\begin{tabular}{||c | c||} 
		\hline
		Gene Activity Pattern & Generation to Add a New Pattern \\ [0.5ex] 
		\hline\hline
		1, -1, 1, -1, 1, -1, 1, -1, 1, -1 & 0 \\ 
		\hline
		1, -1, 1, -1, 1, 1, -1, 1, -1, 1 & 500 \\
		\hline
	\end{tabular}
\end{table}

%\begin{table}[h!]
%	\centering
%	\caption{Parameters of the evolutionary simulation}
%	\label{table:X}
%	\begin{tabular}{||c | c | c | c | c | c | c | c | c||} 
%		\hline
%		Edge Size & Perturbation Number & Perturbation Rate & Mutation Rate & Population Size & Tournament Size & Reproduction Rate & Maximum Generation & Elite Number \\
%		\hline\hline
%		20 & 75 & 0.15 & 0.05 & 100 & Proportional & 0.9 & 2000 & 0 or 10 \\ 
%		\hline
%	\end{tabular}
%\end{table}

\subsection{Diagonal Crossover Mechanism Promotes Modularity}
I simulated 40 independent evolutions for the development with no crossover and with each of the two crossover mechanisms, namely horizontal crossover and diagonal crossover, respectively. None of these simulations applied elitism. Overall, the diagonal crossover mechanism performed better than no crossover and the horizontal crossover, regarding both regulatory performance and modularity emergence, as Tables 3.1 and 3.2 indicate.

The Boolean model that I have utilised to simulate biological networks was originally proposed by Wagner in his study on "epigenetic stability" \cite{wagner1996does}. His work indicated that random recombination made no difference for the evolution of stability, which may be due to the freeness of random recombination on choosing locations to undertake crossover. This can corrupt the modular structures in biological networks.

Conversely, my experimental results suggested that proper recombination methods can contribute to the evolvability of organisms. The diagonal crossover proposed in this report is able to preserve underlying network modules. Although the crossover mechanism utilised by Larson et al. did not preserve community structures as well as diagonal crossover, its partitioning is still based on a network-like structure. This can be the reason why both of these two crossover mechanisms could help in obtaining modularity, with diagonal crossover better than horizontal crossover. Meanwhile, different combinations of parental traits can increase the diversity of the population so that the evolution can be more exploratory.

\subsection{Greed Hampers Modularity}
\subsubsection{Proportional Selection outcompetes Tournament Selection}