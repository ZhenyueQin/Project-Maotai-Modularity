\section{Experiments}
Gene activity patterns and the essential parameters of our evolutionary simulations are provided in the form of Tables 4.1 and 4.2 in order to facilitate repeatability of these experiments. The detailed explanations of these parameters are given after Table 4.2. Overall, only the elite number and the tournament size will be specified in each experiment, since only those may vary in different experiments. All the other parameters are specified in Table 4.1 and 4.2 are consistent in the experiments. 
\vspace*{-12 pt}
\begin{table}[h]
	\centering
	\caption{Table to test captions and labels}
	\label{table:4.1}
	\begin{tabular}{|c | c|} 
		\hline
		Gene Activity Pattern & Generation to Add a New Pattern \\ [0.5ex] 
		\hline
		1, -1, 1, -1, 1, -1, 1, -1, 1, -1 & 0 \\ 
		\hline
		1, -1, 1, -1, 1, 1, -1, 1, -1, 1 & 500 \\
		\hline
	\end{tabular}
\end{table}

\vspace*{-19.5 pt}
\begin{table}[h]
	\centering
	\caption{Parameters of the evolutionary simulation}
	\label{table:4.2}
	\begin{tabular}{|c | c | c |} 
		\hline
		Edge Size & Perturbation Number & Perturbation Rate \\
		\hline
		20 & 75 & 0.15 \\
		\hline
		Mutation Rate & Population Size & Tournament Size \\
		\hline
		0.05 & 100 & Proportional \\
		\hline
		Reproduction Rate & Maximum Generation & Elite Number \\
		\hline
		0.9 & 2000 & 0 or 10 \\
		\hline
	\end{tabular}
\end{table}

\begin{table}[!htbp]
	\centering
	\caption{Explanations of simulation parameters}
	\label{table:4.3}
	\begin{tabular}{|p{0.3\linewidth} | p{0.7\linewidth}|}
		\hline
		Gene Activity Patterns & the patterns that are perturbed, and towards which gene regulatory networks evolve. \\
		\hline
		Generations to add a new pattern & the generations to add new gene activity patterns towards which networks evolve. \\
		\hline
		Edge Size & the initial number of edges in the original gene regulatory networks. \\
		\hline
		Perturbation Number & the number of corrupted versions of each gene activity pattern. \\
		\hline
		Perturbation Rate & the expectation of the number of corrupted genes in a pattern. \\
		\hline
		Mutation Rate & the probability of a gene node to gain or lose an interaction in a network. \\
		\hline
		Population Size & the number of individuals in the population in every generation. \\
		\hline
		Tournament Size & the size of the tournament selection; where tournament selection is used, the size of the tournament; where proportional sections is used, it is annotated as "proportional". \\
		\hline
		Reproduction Rate & the proportion of children reproduced over the entire population. Any vacancy will be filled by the tournament scheme selecting individuals from the previous generation. \\
		\hline
		Maximum generation & the generation when the simulation will terminate after reaching it. \\
		\hline
	\end{tabular}
\end{table}

\subsection{Diagonal Crossover Mechanism Promotes Modularity}
I simulated 40 independent evolutions for the development with no crossover and with each of the two crossover mechanisms, namely horizontal crossover and diagonal crossover, respectively. None of these simulations applied elitism. Overall, the diagonal crossover mechanism performed better than no crossover and the horizontal crossover, regarding both regulatory performance and modularity emergence, as Tables 3.1 and 3.2 indicate.

The Boolean model that I have utilised to simulate biological networks was originally proposed by Wagner in his study on "epigenetic stability" \cite{wagner1996does}. His work indicated that random recombination made no difference for the evolution of stability, which may be due to the freeness of random recombination on choosing locations to undertake crossover. This can corrupt the modular structures in biological networks.

Conversely, my experimental results suggested that proper recombination methods can contribute to the evolvability of organisms. The diagonal crossover proposed in this report is able to preserve underlying network modules. Although the crossover mechanism utilised by Larson et al. did not preserve community structures as well as diagonal crossover, its partitioning is still based on a network-like structure. This can be the reason why both of these two crossover mechanisms could help in obtaining modularity, with diagonal crossover better than horizontal crossover. Meanwhile, different combinations of parental traits can increase the diversity of the population so that the evolution can be more exploratory.

\subsection{Greed Hampers Modularity}
\subsubsection{Proportional Exceeds Tournament Selection on Generating Modularity}~\\
We simulated 
\subsubsection{Elitism Hampers Modularity}~\\
We simulated 40 evolutionary trials with 10 elites and without any elites. That was 80 trials in total. The experimental results indicate that elitism will hamper both the networks' regulatory capabilities and modularity emergence, as shown in Table 3.3 and Table 3.4.