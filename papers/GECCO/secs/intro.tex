\section{Introduction}
Adaptability is an essential problem for evolutionary algorithms to solve. In other words, what can we do to facilitate engineered robots to evolve in order to adapt themselves to the constantly changing environments, just like biological organisms? Studies have indicated that the lack of modularity is one of the reasons that account for the incapability of artificial biological systems for adapting into and scaling up to higher complexity \cite{kashtan2005spontaneous}. For example, artificial neural networks are assumed to be densely connected, whereas human brains exhibit modular components taking different responsibilities, such as hippocampus for dealing with novel situations and amygdala for emotional controls. As such, it is important to understand the conditions under which modularity spontaneously emerged in biology. Afterwards, engineers may leverage these conditions to design modular systems that can solve more complex problems and are able to autonomously adapt themselves to new working environments. One good example of the module-based engineering design is the "high cohesion, low coupling" principle in software engineering \cite{hitz1995measuring}. This understanding in modularity led to a software engineering booming in the current and last century \cite{hitz1995measuring}. 

Specifically, modularity is defined as the divisibility of structures or functions into sub-units that perform autonomously with each other \cite{schlosser2004modularity}. In other words, a module is a group of elements whose associations occur preferentially within the group \cite{espinosa2010specialization}. Furthermore, Many biological activities and structures can be modeled in the form of networks, such as animal brains, signaling pathways, etc. \cite{clune2013evolutionary}. A network is modular if it can be partitioned into highly connected components, and between these components, there are only sparse connections \cite{clune2013evolutionary}. Therefore, elements within a module will demonstrate the tendency of undertaking coherent functions independently from other elements outside of it \cite{espinosa2010specialization} \cite{larson2016recombination}. In biology, such modules exhibit ubiquity \cite{schlosser2004modularity}. Specifically, they appear at various levels of biological organizations \cite{espinosa2010specialization}. Modularity can promote the evolvability of organisms, where evolvability is defined as the capability of rapidly adapting to novel environments \cite{pigliucci2008evolvability}. Two reasons can justify this statement. Firstly, a modular network may allow changes in a module without disturbing other modules; Secondly, modular structures can be reutilized and combined in different ways in order to perform new functions \cite{espinosa2010specialization}. 

Despite the fact that modularity has gained research interests for decades \cite{wagner2007road}, there is no consensus on its origin and evolutionary direction in biology \cite{wagner2004role}. Among various scenarios to explain the condition under which modularity emerges, two stand out, because their proposed conditions are commonly encountered in nature \cite{wagner2007road}. These two scenarios include modularity-varying evolutionary goals \cite{kashtan2005spontaneous} and specializations in gene activity patterns \cite{espinosa2010specialization}. Specifically, the former states that modular changes in environments may impose an impetus in the emergence of modularity \cite{kashtan2005spontaneous}. That is, organisms that live in the environment whose sub-components are repeatedly and constantly altered demonstrate higher-level modularity than those living in the stable environment. This explanation is plausible due to the ubiquity of fluctuations in the environment \cite{espinosa2010specialization}. However, despite the fact that environments are continuously changing, it is unclear to what extent they vary modularly  \cite{espinosa2010specialization}. 

Espinosa-Soto and Wagner studied the conditions under which gene regulatory networks started exhibiting modular structures \cite{espinosa2010specialization}. They concluded that modularity could arise as a by-product of gene specializations when gene regulatory networks acquire the ability to regulate towards multiple different patterns. Specifically, the distinct sub-components in the regulatory network to regulate sharing and different gene activity patterns will hamper each other's performance. Thus, modular networks that favor fewer connections between modules of the network will break the pleiotropic effect of regulating sharing and distinct gene activity patterns. Moreover, additional gene activity patterns can further improve the modularity. Their work is persuasive since the phenomena that gene regulatory networks acquire new gene activity patterns is ubiquitous in evolution. To be more specific, the same collection of genes exhibits different activity patterns at different phases of development or different locations in organisms \cite{espinosa2010specialization}. Their theory can also act as an alternative explanation of why modular-varying environments result in modularity since organisms need to express different gene patterns for different environments \cite{kashtan2005spontaneous} \cite{espinosa2010specialization}. However, the experiments of Espinosa-Soto and Wagner lacked the crossover phase in their evolutionary simulations. Biologically, crossover is necessary.

In this paper, we aim to investigate the plausibility of the theory stating that gene specialization drives modularity of organisms \cite{espinosa2010specialization}. We will first explore whether there exist methods that can expedite the evolutionary process. For example, crossover is assumed to be an effective method to enhance the efficacy of combining useful traits in evolutionary simulations. Therefore, it is beneficial to explore whether there exists a crossover mechanism that can promote modularity. Similarly, the elitism, which is another common mechanism utilized in the artificial evolution, is also worthwhile exploring its contribution to the computational evolution. Furthermore, will different fitness evaluation methods give rise to different modularity levels?

Moreover, experiments in \cite{espinosa2010specialization} did not demonstrate whether structures with high modularity has gained a dominant status on survivability. In biology, there is no organism that exhibits non-modular structures. As such, one may assume non-modular creatures have been extinct. Therefore, modular individuals are expected to have far better performance than non-modular ones, especially for complicated environments. As such, within the surviving simulated organisms in the gene specialization experiments, we will investigate the dominant status of modularity on survivability by comparing the fitness values of the eminent modular organisms to less modular ones.

Additionally, we also wish to discover what properties of modular structures will obtain in a long-term evolution. That is, towards what direction is the system with high modularity evolving? Although the experiments suggested a significant emergence of modular structures by gene specialization, they only reveal specialization is the origin of modular structures. It did not explain the evolutionary direction of modular systems. 