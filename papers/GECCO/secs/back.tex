\section{Background}
This is where we give more detail of the background in modularity.

We might also put in a brief description of our initial experiments with tournament selection, explaining that we were puzzled why we didn't see any modular solutions this way.

In the inception phase of this project, we utilised the Louvain heuristics to compute the partition of the network vertices in order to maximize the modularity of the given graph \cite{blondel2008louvain}. We applied the tournament selection scheme with the tournament size being three and the elitism mechanism with ten elites in every generation. As a result of this setting, the partition of the gene regulatory networks by the Louvain heuristics demonstrated a very low modularity score. As Figure X indicates, by simulating the work in \cite{espinosa2010specialization}, we had expected there would be a spike after 500 generations on modularity. In contrast, we observed a modularity decrease as a result. 

Figure X. An example of evolutions that did not evolve out high modularity


In order to understand this puzzling phenomenon, we removed the elitism mechanism and changed the tournament to proportional selection scheme. In consequence, we eliminated the deviant phenomenon as Figure X indicates. Therefore, we hypothesized that the elitism mechanism or the tournament selection scheme hamper the evolutionary process on evolving out modular structures. 

