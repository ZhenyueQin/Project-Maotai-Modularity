\begin{abstract}
Understanding biological organisms better can assist in solving complex engineering problems by applying their desirable characteristics and structures. A long-standing biological question is how organisms can quickly adapt themselves to new environments that are constantly changing, which is called evolvability. A key aspect to understand evolvability is to know the origin of modularity. Although various theories have been proposed to explain the conditions under which modularity arises, there is no consensus. In the computational biology, one prevalent theory argues that gene specialisation drives modularity. Our experiments indicated that there existed an inconsistency between this theory and observations in biology, regarding the dominant status of modular structures on evolvability. Subsequent experiments also indicated that networks with high fitness could be converted into modular structures by removing inter-module connections while their performance improved. Furthermore, a fluctuant landscape can also promote modularity. 
\end{abstract}