\section{Conclusions}
In summary, we found that the diagonal crossover mechanism can promote the emergence of modularity. In contrast, elitism hampers the rise of modular networks, which indicates that early optimal individuals did not demonstrate high-level modularity. Further experiments also indicated that the theory on the origin of modularity resulting from specialization has limitations on explaining the surviving dominance of modular systems in biology. Furthermore, networks that have high fitness values could demonstrate better performance after converting them into modular structures by removing their inter-module edges. This suggests that modular networks initialized by gene specialization may evolve towards structures requiring a fewer number of connections in total. However, evolutionary simulations could not find these more optimal solutions. Moreover, individuals that live in more fluctuant environments can result in more modular network structures. Therefore, fluctuant landscapes can be essential for modularity evolution. In the future, we will aim to understand the reason why genetic algorithms could not find a path to individuals with better fitness and modularity. Additionally, we would also investigate the correlation between fluctuation of landscapes and the level of modularity. 